\documentclass[a4paper]{article}

\usepackage[utf8]{inputenc}
\usepackage{polski}
\usepackage[polish]{babel}
\usepackage{hyperref}

\title{\textsc{Klaster obliczeniowy}\\Dokumentacja startowa}
\author{Katarzyna Węgiełek\\Marcin Wardziński\\Paweł Własiuk\\Kamil Sienkiewicz (lider)}

\begin{document}
	\maketitle
	
	\section{Metodologia procesu}
	Incremental development \url{http://en.wikipedia.org/wiki/Iterative_and_incremental_development}
	
	Do realizacji projektu klastra obliczeniowego nasz zespół wybrał model przyrostowy. Składa się on z pięciu etapów.
	\begin{itemize}
		\item Na początku powstaje ogólny projekt całego systemu. Nie może być zmieniany w dalszych fazach.
		\item Następnie wybierana jest częściowa funkcjonalność, która będzie realizowana w tej iteracji.
		\item Powstaje szczegółowy projekt tej funkcjonalności (w modelu kaskadowym), a następnie jest ona implementowana.
		\item Za pomocą testów sprawdzana jest poprawność rozwiązania. Następuje prezentacja klientowi (prowadzącemu zajęcia).
		\item Wszystkie kroki z wyjątkiem pierwszego powtarzane są iteracyjnie, dopóki nie zostanie zaimplementowana pełna funkcjonalność.
	\end{itemize}
	
	
	W porównaniu do modelu kaskadowego, przyrostowy oferuje większą elastyczność. Dopuszczalne są pewne zmiany szczegółów w każdej iteracji. Nie trzeba znać wszystkich detali projektu od samego początku. Można też wcześniej obserwować efekty pracy. Z drugiej strony, konieczność realizowania częściowej funkcjonalności może wydłużyć czas realizacji projektu.\\
	
	
	
	Wybór tego modelu wynika po części z narzuconych w regulaminie przedmiotu zasad. Nie wolno zmieniać ogólnego projektu - etap ten został zrealizowany w poprzednim semestrze. Próby zmian mogą również skutkować niespójnością z modułami innych drużyn. W terminach deadlinów należy zaprezentować określoną częściową funkcjonalność - kolejna cecha modelu przyrostowego. Nasza drużyna nie wybrała innych popularnych modeli, np. scrum z powodu niemożliwości codziennych spotkań i codziennego prezentowania przyrostu kodu. 
	\section{Technologie}
	
	\begin{itemize}
		\item Środowisko: C\#.NET
		\item Prezentacja: Windows Presentation Foundation
		\item Bazy danych: EntityFramework (?)
		\item Testowanie: Moq, NUnit/XUnit
		\item Wersjonowanie: git
		\item Serwisy zewnętrzne: GitHub, Trello, TravisCI
	\end{itemize}
	
	\section{Kamienie milowe}	
	\section{Główne zadania}
	\section{Organizacja czasu}
	\section{Struktura projektu}
	
\end{document}