\documentclass[a4paper]{article}

\usepackage[utf8]{inputenc}
\usepackage{polski}
\usepackage[polish]{babel}
\usepackage{hyperref}

\title{\textsc{Klaster obliczeniowy}\\Dokumentacja startowa}
\author{Katarzyna Węgiełek\\Marcin Wardziński\\Paweł Własiuk\\Kamil Sienkiewicz (lider)}

\begin{document}
	\maketitle
	
	\section{Metodologia procesu}
	Incremental development \url{http://en.wikipedia.org/wiki/Iterative_and_incremental_development}
	
	Do realizacji projektu klastra obliczeniowego nasz zespół wybrał model przyrostowy. Składa się on z pięciu etapów.
	\begin{itemize}
		\item Na początku powstaje ogólny projekt całego systemu. Nie może być zmieniany w dalszych fazach.
		\item Następnie wybierana jest częściowa funkcjonalność, która będzie realizowana w danej iteracji.
		\item Powstaje szczegółowy projekt tej funkcjonalności (w modelu kaskadowym), a następnie jest ona implementowana.
		\item Za pomocą testów sprawdzana jest poprawność rozwiązania. Następuje prezentacja klientowi (prowadzącemu zajęcia).
		\item Wszystkie kroki z wyjątkiem pierwszego powtarzane są iteracyjnie, dopóki nie zostanie zaimplementowana pełna funkcjonalność.
	\end{itemize}
	
	
	W porównaniu do modelu kaskadowego, przyrostowy oferuje większą elastyczność. Dopuszczalne są pewne zmiany szczegółów w każdej iteracji. Nie trzeba znać wszystkich detali projektu od samego początku. Można też wcześniej obserwować efekty pracy. Z drugiej strony, konieczność realizowania częściowej funkcjonalności może wydłużyć czas realizacji projektu.\\
	
	
	
	Wybór tego modelu wynika po części z narzuconych w regulaminie przedmiotu zasad. Nie wolno zmieniać ogólnego projektu - etap ten został zrealizowany w poprzednim semestrze. Próby zmian mogą również skutkować niespójnością z modułami innych drużyn. W terminach deadlinów należy zaprezentować określoną częściową funkcjonalność - kolejna cecha modelu przyrostowego. Nasza drużyna nie wybrała innych popularnych modeli, np. scrum z powodu niemożliwości codziennych spotkań i codziennego prezentowania przyrostu kodu. 
	\section{Technologie}
	
	\begin{itemize}
		\item Środowisko: C\# .NET
		\item Prezentacja: Windows Presentation Foundation
		\item Bazy danych: EntityFramework (?)
		\item Testowanie: Moq, NUnit/XUnit
		\item Inne: Autofac
		\item Wersjonowanie: git
		\item Serwisy zewnętrzne: GitHub, Trello, TravisCI
	\end{itemize}
	
	\section{Kamienie milowe}
	Na podstawie dokumentacji i deadline'ów poszczególnych funkcjonalności nasz zespół 
	postawił sobie kilka celów, które chcemy realizować w poszczególnych iteracjach:
	\begin{enumerate}
		\item Komunikacja
		\begin{itemize}
			\item Klaster zna swoją budowę (pomijając backup), poszczególne komponenty systemu potrafią nawiązać połączenie z CS.
		\item Klaster potrafi coś obliczyć wykorzystując wszystkie swoje elementy
		\item Uruchomienie trybu backup
		\end{itemize}
		\item Algorytm DVRP
			\begin{itemize}
				\item Implementacja podstawowej wersji algorytmu
				\item Optymalizacja algorytmu pod względem wydajnościowym
			\end{itemize}		
	\end{enumerate}
	
	\section{Główne zadania}
		Z racji na wybór metodologii działania nasz zespół opracował zadania wymagające realizacji jedynie w najbliższych iteracjach. Po zakończeniu pierwszego etapu zorganizujemy spotkanie, na którym omówimy i podzielimy kolejne zagadnienia wymagające realizacji w projekcie. Poniżej przedstawiamy skrót tego co chcemy zrealizować w 3 pierwszych iteracjach oraz przybliżone czasy ich trwania.
		
	\begin{enumerate}
		\item Podstawowa budowa klastra, mechanizm komunikacji\\
				Czas trwania iteracji: 09.03.2015 - 15.03.2015
		\begin{itemize}
			\item Dwa moduły mogą się połączyć i wymieniać wiadomości tekstowe.
			\item Wygenerowanie klas z XML schemes.
			\item Serwer komunikacyjny i rejestrowanie komponentów.
			\item Analiza XML w tekście, deserializacja do obiektu odpowiedniego typu.
			\item Serwer komunikacyjny musi aktualizować status (życie komponentów).
			\item Deserializacja XML - przypadki specjalne.
		\end{itemize}
		\item Poprawna komunikacja umożliwiająca rozwiązanie pewnego zadania, bez uwzględnienia trybu backup\\
				Czas trwania iteracji: 16.03.2015 - 22.03.2015 
				%może 24\25? ostatnia iteracja wydaje się prostrza, a druga bardzo duża
		\begin{itemize}
			\item System pluginów, rozpoznawanie i rejestrowanie w aplikacji.
			\item Struktura kolejki zadań i algorytm wyboru zadania (również podzadań oraz rozwiązań do scalenia).
			\item Aktualizacja "stanu wiedzy" komponentów w serwerze komunikacyjnym.
			\item TM: Zlecenie podziału przy komunikacji statusowej.
			\item TM: Zgłaszanie wyniku do CS.
			\item CN: Zlecenie obliczeń przy komunikacji statusowej.
			\item CN: Zgłaszanie wyniku do CS.
			\item TM: Zlecenie złożenia wyników cząstkowych.
			\item TM: Zgłaszanie wyniku finalnego do CS.
			\item Client: Zlecanie zadania do obliczenia.
			\item Client: Pobieranie wyników obliczeń.
		\end{itemize}
		\item Tryb backup\\
				Czas trwania iteracji: 23.03.2015 - 29.03.2015
		\begin{itemize}
			\item Kolejka backupowa.
			\item Łączenie się serwera i znajdowanie swojego miejsca w trybie backup.
			\item Synchronizacja danych między serwerami.
		\end{itemize}
	\end{enumerate}	 
		
	\section{Organizacja czasu}
	\section{Struktura projektu}
	
\end{document}



























