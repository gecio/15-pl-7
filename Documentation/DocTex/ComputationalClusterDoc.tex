\documentclass[12pt,a4paper,titlepage]{report}
\usepackage{polski}
\usepackage[utf8]{inputenc}
\usepackage{amsmath}
\usepackage{amsfonts}
\usepackage{amssymb}
\author{Katarzyna Węgiełek \\ Paweł Własiuk \\ Kamil Sienkiewicz\\ Marcin Wardziński}
\title{\textbf{Computational Cluster}}
\linespread{1.125}
\begin{document}
\maketitle
\tableofcontents

	\chapter{Diagramy Przypadków Użycia}
		\section{Konfiguracja}
			\subsection{Serwer Komunikacyjny}
			\subsection{Menadżer Zadań}
			\subsection{Węzeł obliczeniowy}
			\subsection{Klient}
	\chapter{Diagramy Aktywności}
		\section{Zlecenie rozwiązania problemu}
		\verb+tuDiagram1+
		\textbf{Aplikacja kliencka} nawiązuje połączenie z serwerem głównym, używając adresu IP serwera zapisanego w pliku konfiguracyjnym. Wysyła zapytanie o metadane - adresy IP serwerów backup'owych oraz nazwy klas problemów, które mogą być rozwiązane przez ten klaster obliczeniowy (takie, że istnieje przynajmniej jeden \textit{menadżer zadań} potrafiący obsłużyć dany typ problemu). Serwer przesyła metadane do klienta. Następnie użytkownik wybiera spośród dostępnych nazw klas problemów typ zadania jakie ma zostać rozwiązane i wprowadza do programu wszystkie potrzebne dane wejściowe. Aplikacja kliencka wysyła do serwera zlecenie rozwiązania problemu i podane przez użytkownika dane w formacie XML.
Serwer odbiera zlecenie przysłane przez klienta i umieszcza problem w kolejce problemów danego typu oczekujących na rozwiązanie. Zadanie znajduje się w kolejce, dopóki któryś z \textit{menadżerów zadań}, potrafiących rozwiązać problem tej klasy, nie zakończy obliczeń i nie będzie mógł się nim zająć.
		\section{Odczytanie wyniku}
		\verb+tuDiagram2+
		Po otrzymaniu ostatecznego rozwiązania od \textit{menadżera zadań}, serwer umieszcza problem na liście ukończonych, ale jeszcze nieodczytanych rozwiązań. Następnie użytkownik wybiera z listy problem, którego rozwiązanie chce zobaczyć. Aplikacja kliencka wysyła do serwera żądanie pobrania wskazanego wyniku. Serwer wysyła rozwiązanie odpowiedniego problemu do klienta i usuwa je z listy ukończonych zadań. Klient odbiera wyniki i wyświetla je użytkownikowi.
		\section{Węzeł obliczeniowy}
		Węzeł obliczeniowy przy uruchomieniu zgłasza swoją obecność serwerowi komunikacyjnemu. Informacje temat serwera 
znajdują się w pliku konfiguracyjnym węzła. Węzeł wysyła do serwera informacje na temat typów problemów które jest w stanie 
rozwiązać, dzięki temu serwer komunikacyjny może uwzględniać go przy przesyłaniu do \textbf{Menadżera Zadań} informacji na temat
ilości węzłów potrafiących rozwiązać dany typ problemu. \linebreak
Węzeł obliczeniowy nie wykonujący w danym momencie obliczeń otrzymuje od serwera skolejkowany podproblem, podzielony 
na części przez \textbf{Menadżer Zadań}. Zadaniem węzła obliczeniowego jest rozwiązanie otrzymanego zadania oraz przesłanie
rozwiązania do serwera. Jeżeli podczas wykonywania obliczeń wystąpi błąd, węzeł ma za zadanie przygotować raport o 
błędach a następnie wysłać go do serwera.

\verb+tuDiagram+
\end{document}