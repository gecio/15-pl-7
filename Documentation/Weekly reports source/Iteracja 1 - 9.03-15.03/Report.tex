\documentclass[a4paper]{article}
\usepackage[utf8]{inputenc}
\usepackage{polski}
\usepackage[polish]{babel}

\title{Raport tygodniowy - Iteracja 1}
\date{9 marca - 15 marca}
\author{Grupa 7}

\begin{document}
\maketitle

\section{Zadania wykonane}
\subsection{Podstawowy moduł połączeniowy TCP/IP}
\begin{description}
    \item[Wykonawcy zadania] \hfill \\ Kamil Sienkiewicz, Marcin Wardziński
    \item[Czas poświęciony zadaniu] \hfill \\ 5 godzin
    \item[Krótki opis] \hfill \\
        Wykonano łatwo rozszerzalny moduł do wymiany wiadomości między \texttt{INetServer}em oraz \texttt{INetClient}ami, który wykorzystuje mechanizmy rejestracji do ustawiania odbiorców wiadomości, którzy z kolei są tworzeni w razie potrzeby.
        
\end{description}

\subsection{Wygenerowanie klas z XML}
\begin{description}
    \item[Wykonawcy zadania] \hfill \\ Katarzyna Węgiełek, Paweł Własiuk
    \item[Czas poświęciony zadaniu] \hfill \\ 1 godzina
    \item[Krótki opis] \hfill \\ Wygenerowanie klas na podstawie dostarczonych plików \texttt{XSD}. 
\end{description}

\subsection{Deserializacja XML}
\begin{description}
    \item[Wykonawcy zadania] \hfill \\ Katarzyna Węgiełek
    \item[Czas poświęciony zadaniu] \hfill \\ 2 godziny
    \item[Krótki opis] \hfill \\ Deserializacja dostarczonych danych w formie ciągu tekstu w formacie \texttt{XML} do wcześniej wygenerowanych klas. Podpięcie do mechanizmu wykorzystywanego przez moduł \texttt{Net}.
\end{description}
   
\end{document}