\documentclass[a4paper]{article}
\usepackage[utf8]{inputenc}
\usepackage{polski}
\usepackage[polish]{babel}

\title{Raport - Iteracja 3}
\date{23 marca - 30 marca}
\author{Grupa 7}

\begin{document}
\maketitle

\section{Zadania wykonane}
\subsection{Przykładowy \texttt{TaskSolver}}
\begin{description}
    \item[Wykonawcy zadania] \hfill \\ Kamil Sienkiewicz
    \item[Czas poświęciony zadaniu] \hfill \\ 1 godzina
    \item[Krótki opis] \hfill \\ W ramach zadania został zaimplementowany przykładowy \texttt{TaskSolver} obliczający sumę ciągu arytmetycznego. Stworzony \texttt{TaskSolver} wykorzystywany był do testów komunikacji w trakcie implementacji pozostałych komponentów.
\end{description}

\subsection{Mechanizm rejestrujący pluginy w aplikacji}
\begin{description}
    \item[Wykonawcy zadania] \hfill \\ Kamil Sienkiewicz
    \item[Czas poświęciony zadaniu] \hfill \\ 2 godziny
    \item[Krótki opis] \hfill \\ Implementacja mechanizmu wyszukującego i rejestrującego pluginy w \texttt{Comutational Node} i \texttt{Task Manager}ze.
\end{description}

\subsection{Zgłaszanie wyniku do \texttt{Communication Server}a}
\begin{description}
    \item[Wykonawcy zadania] \hfill \\ Katarzyna Węgiełek
    \item[Czas poświęciony zadaniu] \hfill \\ 5 godziny
    \item[Krótki opis] \hfill \\ Implementacja mechanizmu przesyłającego rozwiązanie zleconego zadania od \texttt{Computational Node}'a i \texttt{Task Manager}a do \texttt{Communication Server}a
\end{description}

\subsection{Poprawna obsługa wiadomości \texttt{StatusMessage}}
\begin{description}
    \item[Wykonawcy zadania] \hfill \\ Katarzyna Węgiełek
    \item[Czas poświęciony zadaniu] \hfill \\ 7 godziny
    \item[Krótki opis] \hfill \\ Stworzenie mechanizmu zlecającego odpowiednie zadanie w zależności od pochodzenia \texttt{StatusMessage}. Zadanie składało się z kilku etapów:
    \begin{itemize}
    	\item zlecenie scalenia wyników cząstkowych przez \texttt{Task Manager},
    	\item zlecenie podzielenia zadania otrzymanego of \texttt{Client}a,
    	\item zlecenie rozwiązania podzadania przez \texttt{Computational Node}
    \end{itemize}
\end{description}

\subsection{Zlecenie zadania do obliczenia}
\begin{description}
    \item[Wykonawcy zadania] \hfill \\ Marcin Wardziński
    \item[Czas poświęciony zadaniu] \hfill \\ 3 godziny
    \item[Krótki opis] \hfill \\ Implementacja mechanizmu przesyłającego zadanie do rozwiązania (od \texttt{Client}a do \texttt{Communication Server}a). W treści zadania zostają przesyłane informacje dotyczące typu zadania oraz parametry wejściowe.
\end{description}

\subsection{Przesyłanie wyniku finalnego do \texttt{Communication Server}a}
\begin{description}
    \item[Wykonawcy zadania] \hfill \\ Katarzyna Węgiełek
    \item[Czas poświęciony zadaniu] \hfill \\ 2 godziny
    \item[Krótki opis] \hfill \\ Implementacja mechanizmu scalającego otrzymane wcześniej rozwiązania częściowe i przesłanie rozwiązania końcowego do \texttt{Communication Server}a.
\end{description}

\subsection{Zamiana ETB na Half-closing TCP}
\begin{description}
    \item[Wykonawcy zadania] \hfill \\ Marcin Wardziński
    \item[Czas poświęciony zadaniu] \hfill \\ 3 godziny
    \item[Krótki opis] \hfill \\ Zamiana mechanizmu rozpoznającego koniec odczytywanych danych. Zamiast czekania na znak kończący wiadomości \texttt{ETB} zastosowano \texttt{Half-closing}. Mechanizm ten umożliwił rozróżnienie końca nadawania danych od znaku podziału wiadomości.
\end{description} 

\subsection{Pobieranie wyników obliczeń z poziomu \texttt{Client}a}
\begin{description}
    \item[Wykonawcy zadania] \hfill \\ Marcin Wardziński
    \item[Czas poświęciony zadaniu] \hfill \\ 4 godziny
    \item[Krótki opis] \hfill \\ Umożliwienie zlecającemu zadanie pobranie wyników po zakończeniu obliczeń lub, w przypadku gdy obliczenia jeszcze trwają, powiadomienie go o stanie w jakim jest zadanie.
\end{description}

\subsection{Śledzenie rozwiązywalnych problemów w podłączonych komponentach}
\begin{description}
    \item[Wykonawcy zadania] \hfill \\ Kamil Sienkiewicz
    \item[Czas poświęciony zadaniu] \hfill \\ 1 godzina
    \item[Krótki opis] \hfill \\ Aktualizacja informacji o liście zadań, które jest w stanie rozwiązać dany komponent w trakcie działania aplikacji.
\end{description}

\subsection{Struktura kolejki zadań}
\begin{description}
    \item[Wykonawcy zadania] \hfill \\ Kamil Sienkiewicz
    \item[Czas poświęciony zadaniu] \hfill \\ 3 godziny
    \item[Krótki opis] \hfill \\ Stworzenie struktury pozwalającej kolejkować zadania w  \texttt{Communication Server}ze. Implementacja algorytmu umożliwiającego odpowiedni wybór kolejnego zadania (do podziału, rozwiązania luz scalenia).
\end{description}

\subsection{Obsługa łączenia rozwiązań}
\begin{description}
    \item[Wykonawcy zadania] \hfill \\ Marcin Wardziński
    \item[Czas poświęciony zadaniu] \hfill \\ 2 godziny
    \item[Krótki opis] \hfill \\ Implementacja mechanizmu łączenia podzadań przez \texttt{Task Manager} oraz aktualizacja informacji o stanie poszczególnych zadań w serwerze.
\end{description}

\subsection{Przygotowanie aplikacji do pierwszego checkpointa}
\begin{description}
    \item[Wykonawcy zadania] \hfill \\ Katarzyna Węgiełek, Kamil Sienkiewicz, Marcin Wardziński 
    \item[Czas poświęciony zadaniu] \hfill \\ 20 godzin
    \item[Krótki opis] \hfill \\ Przygotowanie aplikacji do uruchomienia w trakcie zajęć. Poprawianie błędów wynikających z testowych uruchomień aplikacji.
\end{description}

\end{document}