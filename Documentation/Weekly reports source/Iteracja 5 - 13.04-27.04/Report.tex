\documentclass[a4paper]{article}
\usepackage[utf8]{inputenc}
\usepackage{polski}
\usepackage[polish]{babel}

\title{Raport - Iteracja 5}
\date{13 kwietnia - 27 kwietnia}
\author{Grupa 7}

\begin{document}
\maketitle

\section{Zmiana członka zespołu}
Za zgodą prowadzących przyjeliśmy do siebie nowego członka - Macieja Głowackiego w zastępstwie za Pawła Własiuka, który zrezygnował z realizacji przedmiotu. Kolejna osoba pozwoli nam mniejszym nakładem pracy wykonywać zaplanowane zadania i ułatwi tworzenie kodu lepszej jakości. Kilkutygodniowa absencja wcześniej wymienionego członka wymusiła na nas obieranie łatwiejszych i mniej wydajnych rozwiązań, po to by wyrobić się w ramach czasowych. Maciej przejmie część wcześniej przydzielonych Pawłowi zadań oraz kilka dodatkowych - głównie utrzymaniowych. Między innymi:
\begin{itemize}
\item Szkielet algorytmu DVRP - wczytanie danych do wewnętrznego formatu oraz zarządzanie przesyłem danych między komponentami
\item Poprawianie błędów i uproszczeń powstałych we wcześniejszych fazach programowania
\end{itemize}

\section{Zadania wykonane}
\subsection{Wprowadzenie nowego członka zespołu w projekt}
\begin{description}
    \item[Wykonawcy zadania] \hfill \\ Maciej Głowacki, Katarzyna Węgiełek, Kamil Sienkiewicz, Marcin Wardziński
    \item[Czas poświęciony zadaniu] \hfill \\ 4 godziny
    \item[Krótki opis] \hfill \\ W ramach tego zadania zostało wykonane:
    \begin{itemize}
    	\item udzielenie dostępu do repozytorium nowemu członkowi zespołu,
    	\item zapoznanie z obecnym stanem projektu,
    	\item zaprezentowanie rozwiązań zastosowanych w projekcie oraz zapoznanie się z istotnymi fragmentami kodu. 
    \end{itemize}
\end{description}

\subsection{Spotkanie zespołu}
\begin{description}
    \item[Wykonawcy zadania] \hfill \\ Maciej Głowacki, Katarzyna Węgiełek, Kamil Sienkiewicz, Marcin Wardziński
    \item[Czas poświęciony zadaniu] \hfill \\ 6 godziny
    \item[Krótki opis] \hfill \\ W takcie spotkania nastąpiła drobna reorganizacja prac z racji dołączenia nowego członka zespołu.
\end{description}

\subsection{Praca z wieloma \texttt{Computational Node}'ami}
\begin{description}
    \item[Wykonawcy zadania] \hfill \\ Marcin Wardziński
    \item[Czas poświęciony zadaniu] \hfill \\ 3 godziny
    \item[Krótki opis] \hfill \\ Implementacja poprawki umożliwiającej współpracę klastra z wieloma \texttt{Computational Node}'ami.
\end{description}

\subsection{Obsługa \texttt{ErrorMessage}}
\begin{description}
    \item[Wykonawcy zadania] \hfill \\ Katarzyna Węgiełek
    \item[Czas poświęciony zadaniu] \hfill \\ 4 godziny
    \item[Krótki opis] \hfill \\ Dodanie do aplikacji \texttt{ErrorMessage}, implementacja odpowiedniej obsługi dla tej wiadomości.
\end{description}
   
\subsection{Obsługa parametrów}
\begin{description}
    \item[Wykonawcy zadania] \hfill \\ Kamil Sienkiewicz
    \item[Czas poświęciony zadaniu] \hfill \\ 4 godziny
    \item[Krótki opis] \hfill \\ Implementacja paresera umożliwiającego poprawną interpretację i obsługę parametrów poszczególnych aplikacji.
\end{description}

\subsection{Przesyłanie \texttt{NoOperationMessage} do \texttt{Client}a}
\begin{description}
    \item[Wykonawcy zadania] \hfill \\ Katarzyna Węgiełek
    \item[Czas poświęciony zadaniu] \hfill \\ 1,5 godziny
    \item[Krótki opis] \hfill \\ Przesyłanie do \texttt{Client}a \texttt{NoOperationMessage} zawierającą informację o \texttt{backup Server}ze. Wiadomość odsyłana jest do \texttt{Client}a w odpowiedzi na wiadomość zlecającą zadanie do rozwiązania.
\end{description}
   
\subsection{Wielowątkowość komponentów}
\begin{description}
    \item[Wykonawcy zadania] \hfill \\ Marcin Wardziński
    \item[Czas poświęciony zadaniu] \hfill \\ 1 godzina
    \item[Krótki opis] \hfill \\ Poprawa mechanizmu wysyłania zadań do wielowątkowych komponentów.
\end{description}   

\subsection{Obsługa odłączenia komponentów}
\begin{description}
    \item[Wykonawcy zadania] \hfill \\ Marcin Wardziński
    \item[Czas poświęciony zadaniu] \hfill \\ 2 godzina
    \item[Krótki opis] \hfill \\ Implementacja mechanizmu, który zapewnia rozwiązanie każdego zadania. W przypadku, gdy komponent rozwiązujący zadania zostanie wyłączony, zadanie wykonywane przez niego zostanie przydzielone kolejny raz.
\end{description}      
   
\end{document}