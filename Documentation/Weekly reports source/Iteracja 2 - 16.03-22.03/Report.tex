\documentclass[a4paper]{article}
\usepackage[utf8]{inputenc}
\usepackage{polski}
\usepackage[polish]{babel}

\title{Raport tygodniowy - Iteracja 2}
\date{16 marca - 22 marca}
\author{Grupa 7}

\begin{document}
\maketitle

\section{Zadania wykonane}
\subsection{Szkielet \texttt{Communication Server}a. Wstępna obsługa wiadomości \texttt{Register} oraz \texttt{Status}}
\begin{description}
    \item[Wykonawcy zadania] \hfill \\ Kamil Sienkiewicz
    \item[Czas poświęciony zadaniu] \hfill \\ 5 godziny
    \item[Krótki opis] \hfill \\ W ramach zadania został zbudowany framework dla \texttt{Communication Server}a, został skonfigurowany kontener do Dependency Injection oraz podłączono dwóch podstawowych konsumentów wiadomości: \texttt{Register} i \texttt{Status}, serwer prowadzi i aktualizuje listę obecnie podłączonych elementów do klastra.
\end{description}

\subsection{Szkielet \texttt{Client}a }
\begin{description}
    \item[Wykonawcy zadania] \hfill \\ Marcin Wardziński
    \item[Czas poświęciony zadaniu] \hfill \\ 2 godziny
    \item[Krótki opis] \hfill \\ Zbudowanie szkieletu \texttt{Computational Client}a. Rozpoznawanie wejścia z linii poleceń.
\end{description}

\subsection{Szkielet bazy danych}
\begin{description}
    \item[Wykonawcy zadania] \hfill \\ Marcin Wardziński
    \item[Czas poświęciony zadaniu] \hfill \\ 2 godziny
    \item[Krótki opis] \hfill \\ Deserializacja dostarczonych danych w formie ciągu tekstu w formacie \texttt{XML} do wcześniej wygenerowanych klas. Podpięcie do mechanizmu wykorzystywanego przez moduł \texttt{Net}.
\end{description}

\subsection{Obsługa \texttt{SolveRequest}}
\begin{description}
    \item[Wykonawcy zadania] \hfill \\ Marcin Wardziński
    \item[Czas poświęciony zadaniu] \hfill \\ 2 godzin
    \item[Krótki opis] \hfill \\ Podstawowa obsługa \texttt{SolveRequest}, próba integracji z bazą danych.
\end{description}
   
\end{document}