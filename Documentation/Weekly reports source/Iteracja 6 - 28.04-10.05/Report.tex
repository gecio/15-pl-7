\documentclass[a4paper]{article}
\usepackage[utf8]{inputenc}
\usepackage{polski}
\usepackage[polish]{babel}

\title{Raport - Iteracja 6}
\date{28 kwietnia - 10 maja}
\author{Grupa 7}

\begin{document}
\maketitle

\section{Zadania wykonane}
\subsection{Wczytywanie konfiguracji DVRP}
\begin{description}
    \item[Wykonawcy zadania] \hfill \\ Maciej Głowacki
    \item[Czas poświęciony zadaniu] \hfill \\ 6 godzin
    \item[Krótki opis] \hfill \\ Implementacja mechanizmu wczytywania konfiguracji zadania DVRP z pliku umożliwiającego dalsze przetwarzanie danych.
\end{description}

\subsection{Algorytm DVRP}
\begin{description}
    \item[Wykonawcy zadania] \hfill \\ Kamil Sienkiewicz
    \item[Czas poświęciony zadaniu] \hfill \\ 10 godzin
    \item[Krótki opis] \hfill \\ Implementacja algorytmu DVRP w podstawowej wersji, jedynie z drobnymi usprawnieniami. Przeprowadzenie testów na niewielkich problemach.
\end{description}

\subsection{DVRP \texttt{TaskSolver}}
\begin{description}
    \item[Wykonawcy zadania] \hfill \\ Katarzyna Węgiełek, Marcin Wardziński
    \item[Czas poświęciony zadaniu] \hfill \\ 6 godzin
    \item[Krótki opis] \hfill \\ Podpięcie algorytmu \texttt{DVRP} pod \texttt{TaskSolver} oraz wykonanie podstawowych testów obliczenia zadania na klastrze obliczeniowym.
\end{description}
   
\subsection{Rozpoznanie tematu \texttt{Backup Server}a}
\begin{description}
    \item[Wykonawcy zadania] \hfill \\ Marcin Wardziński
    \item[Czas poświęciony zadaniu] \hfill \\ 6 godzin
    \item[Krótki opis] \hfill \\ Analiza zagadnienia implementacji trybu \texttt{backup} dla \texttt{Server}a. Dokonanie koniecznych zmian do uruchomienia \texttt{Server}a w trybie \texttt{backup}. Rejestracja jako \texttt{Backup Server}
\end{description}   

\end{document}